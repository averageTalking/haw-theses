\section{Targeted} \label{sec:targeted}
    Even after eleminating cache effects, a fundamental challenge in memory access remains: the 
    fixed mapping between physical addresses and the underlying \gls{dram} layout. It is insufficient 
    to merely perform \gls{dram} accesses. Rather, precise access to specific \gls{dram} rows is required to
    induce bit flips in adjacent victim rows, denoted as P2. Targeted. The physical address is 
    composed of bits that are partitioned into rank, channel, bank, row, and column, which 
    collectively determine the location where the data is stored [94].
    However, the details of this mapping are not publicly disclosed. Knowledge of the \gls{dram} 
    address mapping enables a wide range of applications, including application-aware memory 
    channel partitioning, adapted page sizes for improved row buffer utilization, efficient use 
    of emerging hybrid memory technologies, evaluation of the effects of unreliable memory, and 
    \gls{dram} layout-aware memory allocators [94].

    \subsection{Reverse Engineering} \label{subsec:reverse-engineering}
        When the \gls{dram} mapping is known, memory systems become more susceptible to attacks, leading 
        to higher pressure and an increased likelihood of bit flips. The addressing scheme can vary 
        for each system, as it depends on multiple factors such as the processor model, DIMM 
        population on the motherboard, and BIOS settings, according to [94]. Therefore, 
        it is necessary to reverse-engineer the \gls{dram} address mapping for all test devices. This 
        process leverages the fact that \gls{dram} consistently follows a hierarchical organization, as 
        described in the \hyperref[sec:background]{Background} section of this thesis. As a consequence of this 
        structure, memory access latencies vary depending on the physical location of data within 
        the hardware [159].

        \subsubsection{Methods} \label{subsubsec:methods}
            \textbf{PALLOC}, introduced by Yun et al. [75] in 2014, uses a dedicated microbenchmark 
            that traverses a linked list over the physical address space to enforce constant \gls{dram} 
            access while controlling bank usage.
            \\
            \textbf{DRAMA}, proposed by Pessl et al. [73] in 2016, exploits the row-buffer timing side 
            channel, where row conflicts cause increased memory access latency. This approach 
            enables the construction of address sets that share channel, rank, and bank 
            properties by performing exhaustive testing of linear addressing functions with an 
            increasing number of coefficients.
            \\
            \textbf{Reliable RE}, presented by Helm et al. [94] in 2020, extends DRAMA by integrating 
            performance counters, which identifies row hits and misses more reliable 
            during the reverse-engineering process.
            \\
            \textbf{DRAMDig}, introduced by Wang et al. [76] in 2020, builds upon DRAMA by incorporating 
            detailed knowledge of specific \gls{dram} chip geometries and processor microarchitecture, thereby 
            reducing the search space of the exponential-time. This method is employed, for example, by 
            the DRAM MaUT tool [79].
            \\
            \textbf{One Bitflip One Cloud Flop}, described by Xiao et al. [93] in 2016, proposes a graph-based 
            approach to infer physical-to-DRAM mappings. It relies on timing differences caused by 
            row conflicts and, similar to DRAMA, primarily aimed at enabling Rowhammer attacks 
            across virtual machines.
            \\
            Common characteristics of DRAMA-based methods include the clustering of addresses into 
            sets that experience row conflicts or row hits, indicating placement within the same bank. 
            These sets are derived by determining which combinations of address bits identify them, 
            often expressed as linear XOR combinations. This approach underlies several attacks 
            and tools, including [38, 53, 72].
            \\
            \textbf{Knock-Knock}, introduced by Plin et al. [99] 2025, reformulates the identification of 
            \gls{dram} parity masks as a linear algebra problem. Instead of relying on exhaustive search, it 
            focuses on solving the mapping directly through algebraic reasoning.

        \subsubsection{Graph-based Approach} \label{subsubsec:graph-based-approach}
            This is a test.
        
        \subsubsection{DRAMA Approach} \label{subsubsec:drama-approach}
            This is a test.

    \subsection{Verification} \label{subsec:verification}
        This is a test.