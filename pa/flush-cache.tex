\subsection{Flush cache} \label{subsec:flush-cache}
    This method for bypassing the cache is recommended because it explicitly invalidates a specific cache 
    line, ensuring that the next access must be served from \gls{dram}. ARMv8 provides explicit cache maintenance 
    instructions for this purpose, most notably the DC CIVAC instruction [Quote of ~159]. By directly 
    targeting individual cache lines, this approach provides deterministic cache-line invalidation, which 
    guarantees predictable behavior whereby every subsequent access is fetched from \gls{dram}. 
    It should be noted that these instructions are no longer available in an unprivileged context but require 
    privileged execution. However, this restriction is acceptable for our project. Overall, this technique 
    avoids reliance on undocumented cache-replacement policies, thereby increasing experimental reliability, and 
    simplifies the setup by eliminating the need for eviction-set construction and timing-based profiling that 
    are typical of cache eviction based approaches.

    \subsubsection{Instruction Types} \label{subsubsec:instruction-types}
        The original Rowhammer approach targeting the x86 architecture relies on the unprivileged 
        \texttt{CLFLUSH} instruction to flush the data associated with a given address from the cache hierarchy 
        [~Flipping Bits]. While cache management instructions in ARMv7-A, referred to as cache maintenance 
        instructions, are privileged in all ARMv7-A processors, they have been exposed to userspace since 
        ARMv8-A. Consequently, it becomes feasible to apply Rowhammer techniques similar to the original 
        CLFLUSH-based approach on systems using ARMv8-A processors [~Triggering RH].
        \begin{figure}[htbp]
            \centering
            \frame{\includegraphics[width=\linewidth]{img/flush-instructions.pdf}}
            \caption{Flush instructions on x86 and ARM}
            \label{fig:flush-instructions}
        \end{figure}

        In the Intel x86 architecture, the \texttt{CLFLUSH} instruction invalidates the cache line containing the 
        specified address from all levels of the cache hierarchy, including both data and instruction caches [72]. 
        If the cache line is dirty at any level of the hierarchy, meaning it contains modified data, it is written 
        back to memory before being invalidated [151, 150, Cache Attacks and Rowhammer]. Both \texttt{CLFLUSH} and 
        its related instructions are unprivileged and therefore accessible from userspace 
        [150, Cache Attacks and Rowhammer].
        \\
        ARMv8-A defines several cache maintenance instructions with distinct semantics. The DC CIVAC 
        instruction cleans and invalidates a data cache by address to \gls{poc} [152], while DC CVAC 
        cleans a data cache line by address to the \gls{poc} without invalidation [153].
        In addition, ARMv8-A introduces the DC ZVA instruction, which 
        is designed to efficiently zero out a block of memory [~Triggering RH]. This instruction zeros a 
        naturally aligned block of N bytes [166]. As described in [Raspberry Pi 3B+], “The ability to preload the data cache 
        with zero values using the DC ZVA instruction is new in ARMv8-A. Cache line zeroing behaves in a similar fashion 
        to a prefetch, as a way of hinting to the processor that certain addresses are likely to be used in the future. 
        However, a zeroing operation can be much quicker as there is no need to wait for external memory 
        accesses to complete [~Triggering RH]."
        \\
        When using DC ZVA in the context of cache bypassing or Rowhammer-style experiments, a crucial requirement 
        is that the memory blocks being zeroed must not already be cached in the hierarchy before entering the 
        hammering loop. Otherwise, the underlying \gls{dram} will not be accessed; instead, only the cached copies 
        are zeroed and marked as dirty. This behavior is fundamentally different from the use of non-temporal 
        store instructions on x86 and would invalidate the intended experimental assumptions [~Triggering RH].
        The corresponding access pattern and instruction sequence are illustrated in Figure~\ref{lst:dczva}.
        \begin{lstlisting}[language=c,caption=\texttt{DC ZVA Usage}, label=lst:dczva]
put addr1 into X9
put addr2 into X10
invalidate/evict addr1 
invalidate/evict addr2 
for i := 0 to N - 1 do 
  DC ZVA, X9 
  DC ZVA, X1
\end{lstlisting}

    
    \subsubsection{Kernel Module} \label{subsubsec:kernel-module}
        The cache bypass mechanism is further investigated using a Linux kernel module designed for kernel-space 
        latency measurements of memory and cache operations on ARMv8 systems. The primary goal of this module is 
        to determine whether cache hits and cache misses can be statistically distinguished and to visualize this 
        distinction by plotting latency distributions for cache hits versus cache misses.
        \\
        The module uses the \gls{cpu} cycle counter \texttt{pmccntr\_el0} in combination with explicit instruction and data 
        synchronization barriers to measure the latency of individual operations in \gls{cpu} cycles. It implements 
        eight different operation modes, each representing a specific memory access pattern, including store and 
        load operations, cache clean operations using DC CVAC, cache clean and invalidate operations using 
        DC CIVAC, and cache line zeroing using DC ZVA. Each of these modes is further evaluated with and without 
        explicit synchronization, thereby intentionally exposing different ordering and completion behaviors. 
        Not all modes employ full memory barriers (DSB and ISB) after the operation, allowing the impact of 
        missing versus enforced barriers on latency to be analyzed. A baseline mode measures plain memory loads 
        to serve as a reference for comparison against all cache maintenance modes. Each configuration is 
        executed repeatedly and averaged over multiple runs to reduce noise. The resulting latency distributions 
        are written to a file and intended to be plotted as diagrams, enabling a clear comparison between modes 
        and highlighting the influence of synchronization on latency and measurement comparability.
        \\
        % Zero Latency readings
        A notable issue encountered during these experiments is the occurrence of zero latency readings when using 
        cache-bypass kernel modules, despite an active \gls{pmu} and relaxed paranoid 
        settings. This behavior is observed specifically on Raspberry Pi 5 systems, and its underlying cause 
        remains unresolved at this stage.
        \\
        % System Control Register on EL1 (SCTLR\_EL1)
        In contrast to the ARMv7 architecture, ARMv8 defines the \texttt{SCTLR\_EL1} system control register, which includes 
        a bit denoted as \texttt{UCI}. If this bit is set, userspace access is enabled for a specific set of cache and 
        instruction maintenance operations, as illustrated in Figure~\ref{fig:sctlr} [Quote of ~Masterthesis, 145].
        \begin{figure}[htbp]
            \centering
            \frame{\includegraphics[width=\linewidth]{img/sctlr.pdf}}
            \caption{Instructions affected by the \gls{uci} bit in SCTLR\_EL1}
            \label{fig:sctlr}
        \end{figure}

        The \texttt{SCTLR\_EL1} register is a 64-bit system control register that provides top-level control of the system, 
        including its memory system, at both EL1 and EL0 [91]. The \texttt{\gls{uci}} field is located at bit position 26 [91,92] 
        and enables EL0 access to the aforementioned cache maintenance instructions when set to 1, while a value of 
        0 disables EL0 access [92]. In addition, \texttt{SCTLR\_EL1} contains the \texttt{\gls{dze}} field at bit 
        position 14 [91,92]. This bit controls EL0 access to the DC ZVA instruction, where a value of 1 enables 
        execution in userspace and a value of 0 disables it [92].
        \\
        To verify the configuration of these fields, a dedicated kernel module \texttt{sctlr.c} was implemented to read 
        and report the contents of \texttt{SCTLR\_EL1}. The module output confirms that both the \texttt{\gls{uci}} and \texttt{\gls{dze}} bits are set. 
        This result demonstrates that DC CIVAC and DC CVAC are unprivileged and can be executed from userspace, 
        and that DC ZVA is likewise unprivileged and accessible at EL0. These findings are consistent with previous 
        observations reported in [~Triggering RH].
    
    \subsubsection{Userspace} \label{subsubsec:userspace}
        A userspace counterpart to the kernel module experiments is implemented in the program \texttt{flush\_user.c}, 
        located in \texttt{/rowhammer/experiments/cache-bypass}. The goal of this experiment is likewise to determine whether 
        cache hits and cache misses can be statistically distinguished and to visualize the difference by plotting 
        corresponding latency distributions.
        \\
        This experiment evaluates the effectiveness and performance impact of cache maintenance instructions on 
        ARM-based systems when executed from userspace. The objective is to assess whether, and to what extent, 
        specific instructions can bypass or influence cache behavior, and to quantify the associated latency costs. 
        The program measures cycle-accurate execution times using ARM performance counters while applying different 
        low-level memory operations. It executes eight predefined instruction patterns involving store, load, clean, 
        invalidate, and zeroing operations. Each pattern is executed repeatedly under two conditions: with and 
        without the corresponding cache maintenance instruction.
        \\
        Compared to the kernel-space implementation, several differences arise. The kernel-based \texttt{flush\_kernel.c} 
        executes in kernel space as a loadable module using \texttt{module\_init} and \texttt{module\_exit}, providing direct access 
        to hardware resources and privileged instructions. In contrast, \texttt{flush\_user.c} runs entirely in user space 
        as a standard program with a \texttt{main()} function and without kernel privileges. Memory management also differs, 
        as the kernel module relies on \texttt{kmalloc()} and \texttt{kfree()} to obtain physically contiguous memory, 
        whereas the userspace program uses standard virtual memory allocation mechanisms such as \texttt{malloc()} and \texttt{free()}. 
        File I/O and logging are handled using kernel-specific interfaces in kernel space, while the userspace implementation 
        uses standard C library functions.
