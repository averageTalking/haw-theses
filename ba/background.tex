\section{Background}
  \subsection{CPU Caches}
    Dynamic Random-Access Memory (DRAM) requires regular refreshing because it stores each bit 
    as an electrical charge, which gradually leaks over time. This leakage necessitates refresh 
    cycles, resulting in slower access times; however, DRAM provides high storage density at a 
    relatively low cost [27]. In contrast, Static Random-Access Memory (SRAM) maintains each bit 
    in a stable flip-flop circuit, eliminating the need for refresh operations. This 
    characteristic makes SRAM faster and allows it to be physically located closer to the CPU, 
    though it is more expensive and offers lower storage capacity compared to DRAM [27]. \\
    CPU caches are constructed from SRAM to maximize access speed [45, 31]. A cache is a fast, 
    specialized memory area designed to accelerate data retrieval [27]. It temporarily stores 
    copies of frequently used data and instructions from main memory (RAM) that the CPU is likely 
    to access imminently [27, 45]. By holding data that the CPU may repeatedly access, caches 
    serve as a buffer between the CPU and main memory, significantly reducing the time required 
    for repeated data accesses [30, 45]. Cached data typically mirrors the contents of slower 
    storage devices such as RAM or disks [27]. Common types of caches include CPU caches, memory 
    caches, and browser caches [27]; however, only CPU caches are relevant in the context of the 
    Rowhammer attack. \\
    Cache performance is characterized by hits and misses. A cache hit occurs when the CPU 
    requests data already present in one of its cache levels. In this case, the cache provides 
    the data immediately, bypassing the need to access main memory. This results in extremely 
    low latency, as cache access is several orders of magnitude faster than DRAM access. 
    Conversely, a cache miss happens when the requested data is not stored in the cache. The 
    CPU must then retrieve the data from DRAM, which introduces significantly higher latency 
    due to slower access times. Once retrieved, the data is typically placed in the cache to 
    accelerate potential future accesses. Cache hits improve overall system performance by 
    exploiting temporal and spatial locality, ensuring that frequently or recently used data 
    remains close to the CPU. \\
    The memory hierarchy organizes storage components according to their speed, latency, and 
    cost, as illustrated in Figure. Memory Hierarchy. Higher levels of the hierarchy, such as 
    CPU caches, are smaller but faster, while lower levels are larger but significantly slower. 
    Due to their proximity to the CPU and implementation using SRAM, caches provide access 
    speeds approximately 10–100 times faster than DRAM [46].