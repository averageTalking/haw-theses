\subsection{Cache eviction} \label{subsec:cache-eviction}
    Cache eviction refers to the process of deliberately forcing cache lines out of the cache by accessing 
    memory addresses that are mapped to the same cache eviction set. By repeatedly accessing such congruent 
    addresses, an attacker can ensure that a specific target cache line is replaced and 
    therefore removed from the cache.

    In the context of \textbf{GPU cache eviction}, prior work has shown that by cleverly evicting the cache of an 
    integrated \gls{gpu}, access to main memory can be achieved with significantly reduced latency, enabling 
    memory accesses~\cite{paperPwning}. This approach exploits the relatively simple cache structures 
    of integrated \gls{gpu}s, which can be evicted efficiently in order to achieve main-memory accesses~\cite{paperOpenGL, paperPwning}.

    \textbf{CPU cache eviction} is considerably more complex~\cite{paperARMageddon}. This complexity arises from higher cache associativity, 
    the presence of multiple cache levels, and sophisticated cache replacement policies. Nevertheless, 
    empirical studies demonstrate that \gls{cpu} cache eviction is feasible in practice, provided that precise 
    timing and detailed knowledge of cache set mappings are available~\cite{thesisCacheAttacksandRH, toolRPI3}.

    This section first outlines fundamental aspects of cache organization, which form the basis for 
    understanding eviction sets, including direct-mapped and set-associative cache structures, as well as 
    key concepts of virtual versus physical indexing and tagging. It then discusses cache management, 
    focusing on cache line replacement policies and cache inclusiveness. Finally, the section presents the 
    development of cache eviction mechanisms, covering functionality, implementation details, and eviction 
    strategies.

    \subsubsection{Cache Organization} \label{subsubsec:cache-organization}
        %% Direct-mapped Caches
        In a direct-mapped cache, each main memory entry is mapped to exactly one cache location. As a 
        consequence, multiple distinct memory addresses can also map to the same cache location~\cite{thesisCacheAttacksandRH}.
        \\
        An illustrative example is shown in \hyperref[fig:direct-mapped-cache]{Figure~\ref{fig:direct-mapped-cache}}, which depicts a cache with a total size of 4~KB, 
        consisting of 128 cache locations and a cache line size of 64~bytes. In this configuration, an address 
        is divided into three fields: 6 bits for the block offset, 7 bits for the index, and 19 bits for the 
        tag.
        \begin{figure}[htbp]
            \centering
            \frame{\includegraphics[width=\linewidth]{img/direct-mapped-cache.pdf}}
            \caption{Direct-mapped cache~\cite{thesisCacheAttacksandRH}}
            \label{fig:direct-mapped-cache}
        \end{figure}
        To determine whether a specific address is present in the cache, the index bits of the address are 
        first identified, and the associated cache line’s stored tag is then compared with the address’s tag.
        If the tags match and the valid bit is set, a cache hit occurs, and the block offset is used to select 
        the relevant word from the cache line. If the tags do not match or the valid bit is not set, indicating 
        that the cache entry does not contain valid data, a cache miss occurs, and the cache line is replaced 
        with data fetched from the requested memory address.


        %% Set-Associative Caches
        If the replacement policy is allowed to choose any entry in the cache to hold a the data, the 
        cache is referred to as fully associative~\cite{thesisCacheAttacksandRH}. Modern processors, however, typically employ 
        set-associative caches, which organize cache lines into multiple sets~\cite{thesisCacheAttacksandRH}. In an N-way set-associative 
        cache, each memory block can be placed in any one of N cache lines within a specific set.
        \\
        Each memory address maps to exactly one cache set, and addresses that map to the same set are called 
        congruent addresses, as illustrated in \hyperref[fig:n-way-associative-cache]{Figure~\ref{fig:n-way-associative-cache}}. 
        When all cache lines of a set are occupied and a new block needs to be loaded, the cache replacement 
        policy determines which existing cache line is evicted in order to accommodate the new data~\cite{thesisCacheAttacksandRH}.

        \begin{figure}[htbp]
            \centering
            \frame{\includegraphics[width=\linewidth]{img/n-way-associative-cache.pdf}}
            \caption{N-way associative cache~\cite{thesisCacheAttacksandRH}}
            \label{fig:n-way-associative-cache}
        \end{figure}

        \begin{table}[h!]
            \centering
            \caption{RPi4 Cache: \texttt{lscpu {-}{-}cache}}
            \label{tab:rpi4-cache}
            \resizebox{\textwidth}{!}{%
                \begin{tabular}{lrrrrrrr}
                    \hline
                    NAME & ONE-SIZE & ALL-SIZE & WAYS & TYPE & LEVEL & SETS & COHERENCY-SIZE \\
                    \hline
                    L1d & 32K  & 128K  & 2  & Data        & 1 & 256  & 64 \\
                    L1i & 48K  & 192K  & 3  & Instruction & 1 & 256  & 64 \\
                    L2  & 1M   & 1M    & 16 & Unified     & 2 & 1024 & 64 \\
                    \hline
                \end{tabular}%
            }
        \end{table}
        \begin{table}[h!]
            \centering
            \caption{RPi5 Cache: \texttt{lscpu {-}{-}cache}}
            \label{tab:rpi5-cache}
            \resizebox{\textwidth}{!}{%
                \begin{tabular}{lrrrrrrr}
                    \hline
                    NAME & ONE-SIZE & ALL-SIZE & WAYS & TYPE & LEVEL & SETS & COHERENCY-SIZE \\
                    \hline
                    L1d & 64K  & 256K  & 4  & Data        & 1 & 256  & 64 \\
                    L1i & 64K  & 256K  & 4  & Instruction & 1 & 256  & 64 \\
                    L2  & 512K & 2M    & 8  & Unified     & 2 & 1024 & 64 \\
                    L3  & 2M   & 2M    & 16 & Unified     & 3 & 2048 & 64 \\
                    \hline
                \end{tabular}%
            }
        \end{table}
        \hyperref[tab:rpi4-cache]{Tables \ref{tab:rpi4-cache}} and \ref{tab:rpi5-cache} summarize the cache configurations of the Raspberry 
        Pi 4 and Raspberry Pi 5, respectively. These tables provide 
        an overview of cache sizes, associativity, cache levels, and the number of cache sets for the different 
        cache hierarchies present on both platforms.
        
        The number of cache sets is determined by the cache size, the cache line size, and the set associativity. 
        This relationship can be expressed as
        \[
        2^{\text{Index}} = \frac{\text{Cache Size}}{\text{Line Size} \times \text{Set associativity}}
        \]
        Applying this formula to the L2 cache of the Raspberry Pi 4 results in
        \[
        \text{Number of sets} = \frac{1024 \times 1024}{64 \times 16} = 1024 = 2^{10}
        \]
        Similarly, for the L2 cache of the Raspberry Pi 5, the number of cache sets is given by
        \[
        \text{Number of sets} = \frac{512 \times 1024}{64 \times 8} = 1024 = 2^{10}
        \]
        These calculations show that, despite differences in cache size and associativity, both systems employ 
        the same number of cache sets at the L2 cache level.
        \\

        %% Virtual and Physical Tags and Indexes
        \noindent
        \gls{cpu} caches can be indexed and tagged using either virtual or physical addresses. When virtual 
        addresses are used, it is possible for the same physical address to be cached in multiple cache lines, 
        which can negatively affect performance. Cache lines are uniquely identified by a tag, which may 
        be derived from either the virtual or the physical address~\cite{thesisCacheAttacksandRH}.
        \\
        In \textbf{\gls{vivt}} caches, both the index and the tag are derived from the 
        virtual address. Such caches are generally faster because no address translation is required before 
        lookup; however, the virtual tag is not unique, and shared memory may appear multiple times in the cache.
        \\
        \textbf{\gls{pipt}} caches use the physical address for both index and tag. 
        This approach avoids duplication of shared memory in the cache, but it is slower because address 
        translation via the \gls{tlb} is required before cache access.
        \\
        \textbf{\gls{pivt}} caches use the physical address for indexing and the virtual 
        address for tagging. This design provides no practical benefit, as it still requires address translation 
        while retaining the disadvantage of non-unique virtual tags, allowing shared memory to be duplicated.
        \\
        \textbf{\gls{vipt}} caches use the virtual address for indexing and the physical 
        address for tagging. This approach offers lower latency than \gls{pipt} caches because index lookup can proceed 
        in parallel with \gls{tlb} translation, although tag comparison must still wait for the physical address.
        \\
        On the Raspberry Pi 4, the L1 instruction cache, L1 data cache, and L2 cache behave as \gls{pipt} caches~\cite{webRPi4Hardware, docCortexA72}. 
        On the Raspberry Pi 5, the L1 instruction and L1 data caches are \gls{vipt}, but they behave effectively as \gls{pipt}
        4-way set-associative caches~\cite{docCortexA76}.
    
    \subsubsection{Cache Management} \label{subsubsec:cache-management}
        %% Cache Replacement Policy
        When a cache is full and a cache miss occurs, buffered code and data must be evicted from the cache to make room for new 
        cache entries. The heuristic used to decide which cache entry to evict is referred to as the cache replacement 
        policy, which aims to select the entry that is least likely to be used in the near future. The chosen 
        replacement policy directly influences the number and access patterns required for an effective eviction 
        strategy~\cite{thesisCacheAttacksandRH}.
        \\
        The \gls{lru} replacement policy replaces the cache entry that has not been accessed for 
        the longest time. In contrast, a pseudo-random replacement policy selects a cache entry to evict based 
        on a pseudorandom number generator~\cite{thesisCacheAttacksandRH}.
        \\
        On the Raspberry Pi 4, the L1 instruction and L1 data caches use an \gls{lru} replacement policy, while the L2 
        cache employs a pseudo-\gls{lru} replacement policy~\cite{webRPi4Hardware, docCortexA72}. On the Raspberry Pi 5, the L1 instruction and 
        L1 data caches use a pseudo-\gls{lru} replacement policy, and a dynamic biased replacement policy is 
        employed~\cite{docCortexA76}.
        \\
        %% Inclusiveness
        In a multi-level cache hierarchy, it must be decided which cache levels store copies of data. A 
        higher-level cache is called inclusive with respect to a lower-level cache if all cache lines present in 
        the lower-level cache are also stored in the higher-level cache. Caches are considered exclusive if 
        a cache line can reside in only one of two cache levels~\cite{thesisCacheAttacksandRH}. If a cache is neither inclusive nor exclusive, 
        it is referred to as non-inclusive.
        \\
        On the Raspberry Pi 4, the L2 cache is inclusive with respect to the L1 cache. On the Raspberry Pi 5, the 
        cache hierarchy is strictly inclusive with respect to the L1 data cache and weakly inclusive with respect 
        to the L1 instruction cache~\cite{docCortexA76}. The L3 cache is neither strictly inclusive nor strictly exclusive, but 
        rather non-inclusive, with a dynamic allocation policy that depends on the access pattern~\cite{docSharedUnit, webInclusivness}. 
        Specifically, a first-time access by a core allocates data only to its L1 and L2 caches and not to the 
        L3. If data is evicted from the L2 cache, it can be reloaded into the L3 cache, exhibiting exclusive behavior. 
        If multiple cores share the same cache line, it may reside simultaneously in the L3 cache and in multiple 
        L2 caches, exhibiting inclusive behavior~\cite{docSharedUnit, webInclusivness}.

    \subsubsection{Eviction Development} \label{subsubsec:eviction-development}
        The cache eviction mechanism used in this work is based on the libflush framework and relies on the 
        construction and use of eviction sets. 
        \\
        % Eviction Set Structure
        The eviction set structure, as defined in \hyperref[lst:eviction-set-structure]{Figure~\ref{lst:eviction-set-structure}}, consists of multiple virtual 
        addresses that are congruent, meaning that they map to the same cache set. Accessing all addresses contained 
        in an eviction set therefore ensures that a targeted cache line occupying the same set can be evicted from the cache.
        \begin{lstlisting}[language=c,caption=\texttt{Eviction set structure}, label=lst:eviction-set-structure]
typedef struct congruent_address_cache_entry_s {
  bool used;
  void* congruent_virtual_addresses[ADDRESS_COUNT];
} congruent_address_cache_entry_t;
\end{lstlisting}


        % Eviction Function
        The actual eviction procedure is implemented as a nested loop structure that repeatedly accesses the virtual 
        addresses contained in an eviction set (see \hyperref[lst:eviction-function]{Figure~\ref{lst:eviction-function}}). During execution, 
        all addresses are iterated over multiple times, thereby generating sustained pressure on the corresponding cache set. 
        These repeated memory accesses are intended to reliably force the eviction of a target cache line. The behavior 
        of the eviction routine is governed by several configuration parameters. Together, these 
        parameters control both the repetition frequency of the eviction process and the coverage of memory accesses 
        within the eviction set.
        \begin{lstlisting}[language=c,caption=\texttt{Eviction function}, label=lst:eviction-function]
inline void evict(congruent_address_cache_entry_t* congruent_address_cache_entry) {
	for (unsigned int i = 0; i < ES_EVICTION_COUNTER; i += 1) {
		for (unsigned int j = 0; j < ES_NUMBER_OF_ACCESSES_IN_LOOP; j++) {
			for (unsigned int k = 0; k < ES_DIFFERENT_ADDRESSES_IN_LOOP; k++) {
				libflush_access_memory(congruent_address_cache_entry->congruent_virtual_addresses[i+k]);
			}
		}
	}
}
\end{lstlisting}


        %  Cache Set Determination
        To correctly construct eviction sets, it is necessary to determine the cache set index of a given memory 
        address (see \hyperref[lst:cache-set-determination]{Figure~\ref{lst:cache-set-determination}}). This is achieved by computing the set index from the 
        physical address of the memory location. The calculation is performed by shifting the physical address by 
        \texttt{LINE\_LENGTH\_LOG2} bits, corresponding to the cache line size, and then taking the result modulo 
        the total number of cache sets. This computation ensures that all addresses selected for an eviction set 
        indeed target the same cache set, which is a prerequisite for effective cache eviction.
        \begin{lstlisting}[language=c,caption=\texttt{Cache set determination}, label=lst:cache-set-determination]
size_t libflush_eviction_get_set_index(libflush_session_t* session, void* address) {
	uintptr_t physical_address = libflush_get_physical_address(session, (uintptr_t) address);
	return (physical_address >> LINE_LENGTH_LOG2) % NUMBER_OF_SETS;
}
\end{lstlisting}


        %% Build and Integration Adjustments
        The libflush library was built for the ARMv8 architecture.
        A system-wide installation without Android dependencies was performed. The provided example application was subsequently 
        built and executed
        which additionally generated measurement plots. To streamline the build process, the Makefile in 
        \texttt{armageddon/libflush} was improved by introducing Raspberry Pi specific default build targets, allowing 
        straightforward execution of \texttt{make} and \texttt{make install}. Furthermore, the Makefile in 
        \texttt{cache-bypass} was modified for simplify the usage by forwarding these targets to \texttt{armageddon/libflush}. The \texttt{libflush/example} 
        application was adapted to be directly executable and was initially used to evaluate experimental observations. 
        It was then extended to include averaging over multiple repetitions to improve statistical robustness, automated 
        sequential execution of all cache attack methods, persistent storage of measurement results in \texttt{.dat} 
        files, and a Python-based solution for automated plotting and visualization. Finally, the adapted example was 
        integrated into the local \texttt{cache-bypass} directory structure.
        \\
        % Eviction Strategy
        Modern cache attack techniques such as \textit{Flush+Reload} and \textit{Flush+Flush} rely on the unprivileged x86 flush 
        instruction \texttt{clflush} to evict cache lines. With the exception of ARMv8-A CPUs, ARM processors do 
        not provide an unprivileged flush instruction, which makes cache eviction strategies necessary instead. 
        In order to obtain the high-frequency measurements required by modern cache attacks, eviction must be 
        sufficiently fast, as previously proposed eviction strategies have been shown to be too slow~\cite{paperCacheAccessPattern}. 
        In this work, the cache eviction implementation proposed by Lipp et~al.~\cite{toolArmageddon} is used. This approach is 
        applicable only to devices with a known eviction strategy~\cite{thesisDRAMA}, and therefore the default configuration 
        provided by the libflush tool is adopted (see \hyperref[lst:default]{Figure~\ref{lst:default}}).
        \input{listings/default.h.tex}

        The relevant configuration parameters are defined in the default libflush configuration. The parameter 
        \texttt{NUMBER\_OF\_SETS} specifies the total number of cache sets considered for eviction and determines 
        how addresses are indexed to cache sets. The parameters \texttt{LINE\_LENGTH\_LOG2} and \texttt{LINE\_LENGTH} 
        define the cache line size of 64~bytes and its base-2 logarithm, which are required for address calculations. 
        The parameter \texttt{ES\_EVICTION\_COUNTER} determines the number of addresses in an eviction set that are 
        accessed per iteration in order to reliably evict a target cache line. The parameter 
        \texttt{ES\_NUMBER\_OF\_ACCESSES\_IN\_LOOP} specifies how often the eviction loop is repeated to ensure eviction 
        under noisy conditions. Finally, \texttt{ES\_DIFFERENT\_ADDRESSES\_IN\_LOOP} defines the number of distinct 
        addresses accessed in each inner iteration of the eviction loop, thereby distributing accesses across the 
        cache set to achieve robust eviction behavior.
