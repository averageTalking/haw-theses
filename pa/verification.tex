\subsection{Verification} \label{subsec:verification}
    A \gls{dram} mapping has been obtained for each test device. However, it remains uncertain whether 
    the reverse-engineered mapping is entirely correct (as seen in the 
    \hyperref[subsubsec:bank-function-derivation]{previous part}). Several methods can be employed 
    to verify the correctness of the derived \gls{dram} addressing functions.
    \\
    One approach is \textbf{physical probing}, which involves comparing the determined functions with results 
    obtained through direct hardware measurements, as described in [73]. This method, however, faces 
    significant practical limitations. Embedded systems often incorporate small chips and 
    multi-layered circuit boards [159], making direct probing difficult. In particular, verification 
    via hardware probing is not feasible on devices such as Raspberry Pis due to the presence of 
    soldered \gls{dram} modules.
    \\
    Another verification method leverages \textbf{Rowhammer} testing. In this approach, correctness is 
    assessed by evaluating whether the use of the addressing functions in Rowhammer experiments 
    increases the number of bit flips per second proportionally to the number of sets identified [73]. 
    This method is often impractical because tests may require excessively long durations, the systems 
    under study may not be vulnerable, or mitigations may interfere with the results.
    \\
    The final method is a \textbf{software-based} verification, which entails performing timing measurements 
    across a larger set of addresses to detect differences consistent with the derived addressing 
    functions [73]. This approach is feasible and provides a practical means of validating the 
    correctness of the reverse-engineered \gls{dram} mappings.

    \subsubsection{Software-based} \label{subsubsec:software-based}
        This is a test.

    \subsubsection{Rowhammer} \label{subsubsec:rowhammer}
        This is a test.
