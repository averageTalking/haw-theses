\section{Methodology} \label{sec:methodology}
  This chapter describes the methodological framework applied in this thesis. It outlines the overall 
  research design, the selected test devices, the employed data sources, as well as the procedures for 
  data collection, analysis, and evaluation. Furthermore, it specifies the functional and non-functional 
  requirements that guided the implementation of the experimental setup.

  \subsection{Research Design} \label{subsec:research-design}
    % Research Type && -Objective && Methodological Approach
    The research conducted in this thesis follows a quantitative study design based on experimental 
    testing, with the aim of confirming or refuting predefined hypotheses. The primary research objective 
    is to assess the susceptibility of various \hyperref[tab:experimental-setups]{Raspberry Pi models} to Rowhammer attacks. \\
    From a methodological perspective, the study primarily adopts a deductive approach. Existing 
    Rowhammer theories are tested on Raspberry Pi platforms by starting from established theoretical 
    foundations, formulating hypotheses derived from these theories, and subsequently validating or 
    refuting them through controlled experiments. In general, this deductive approach is characterized by 
    the investigation of an existing theory, followed by the derivation of hypotheses, the execution of 
    experiments, and the confirmation or falsification of the proposed assumptions. \\
    In addition to the deductive core, the methodology incorporates an inductive approach in a supporting 
    role. Experimental observations obtained during testing may reveal unexpected system behavior. Based 
    on these measured results, new hypotheses or insights can be derived. In general terms, the inductive 
    approach involves the derivation of theory from empirical research by identifying patterns in 
    observations and formulating new hypotheses or theoretical considerations. Overall, the study is 
    classified as predominantly deductive, complemented by inductive elements that emerge from experimental 
    findings. 
    \\
    % Test Devices
    \begin{table}[h!]
      \centering
      \caption{Experimental setups}
      \label{tab:experimental-setups}
      \begin{tabular}{|l|l|l|l|l|}
        \hline
        \textbf{Platform} & \textbf{Processor} & \textbf{Architecture} & \textbf{SoC} & \textbf{DRAM} \\ \hline
        Raspberry Pi 4B & ARMv8 & aarch64 & Broadcom BCM2711 & 4 GB LPDDR4 \\ \hline
        Raspberry Pi 4B & ARMv8 & aarch64 & Broadcom BCM2711 & 8 GB LPDDR4 \\ \hline
        Raspberry Pi 5 & ARMv8 & aarch64 & Broadcom BCM2712 & 4 GB LPDDR4x \\ \hline
        Raspberry Pi 5 & ARMv8 & aarch64 & Broadcom BCM2712 & 8 GB LPDDR4x \\ \hline
        Raspberry Pi 5 & ARMv8 & aarch64 & Broadcom BCM2712 & 16 GB LPDDR4x \\ \hline
      \end{tabular}
    \end{table}

    The experimental evaluation is conducted on multiple Raspberry Pi devices, as illustrated in 
    Table~\ref{tab:experimental-setups}. The tested hardware includes Raspberry Pi 4 models with 4 GB and 8 GB of 
    RAM, as well as Raspberry Pi 5 models with 4 GB, 8 GB, and 16 GB of RAM. \\
    The Raspberry Pi 4 is based on a quad-core ARMv8 Cortex-A72 CPU clocked at 1.8 GHz. Each core is equipped
    with an L1 cache consisting of 32 KB data cache and 48 KB instruction cache, and all cores share a 1 MB 
    L2 cache. This configuration makes the Raspberry Pi 4 approximately 50\% faster than the Raspberry Pi 3B 
    [64]. 
    \\
    The Raspberry Pi 5 features a quad-core ARMv8 Cortex-A76 CPU operating at 2.4 GHz. Each core provides a 
    64 KB L1 cache for both data and instructions, a dedicated 512 KB L2 cache, and a shared 2 MB L3 cache. 
    In aggregate, the architectural improvements introduced with the BCM2712 result in a performance uplift 
    of approximately two to three times compared to the Raspberry Pi 4 for common CPU- or I/O-intensive 
    workloads [64]. 
    \\
    % Data Sources
    The study is based on multiple data sources that support both the theoretical foundation and the 
    experimental validation of the research hypotheses. These sources include related academic publications 
    from established conferences such as \textit{ACM} and \textit{IEEE}, as well as technical documentation, datasheets, and 
    glossaries relevant to Raspberry Pi platforms and Rowhammer attacks. In addition, publicly available \textit{GitHub} 
    repositories providing Rowhammer proof-of-concept implementations are used, which are adapted and applied 
    to the target devices. Beyond existing resources, own experiments are implemented and executed specifically 
    for this study in order to verify the formulated hypotheses through direct empirical evaluation. 
    \\
    % Data Collection
    Data collection is performed through the experimental execution of Rowhammer attacks on each Raspberry 
    Pi model. During these experiments, system behavior is observed and relevant outcomes are recorded and 
    logged. The collected data serves as the basis for drawing conclusions from the observed results and for 
    formulating further hypotheses. All data is obtained directly from experiments conducted on the Raspberry 
    Pi devices. 
    \\
    % Data Analysis
    The analysis of the collected data is supported by automated scripts and the generation of plots using 
    Python. Experimental results are logged using multiple formats, including standard \texttt{.log} files, structured 
    JSON logs, and \texttt{.dat} files. This combination of logging mechanisms enables systematic evaluation and 
    comparison of experimental outcomes.

  \subsection{Evaluation} \label{subsec:evaluation}
    To ensure the reliability and validity of the results, experiments are repeated to verify the consistency 
    of observations. All tests are conducted in a controlled environment to minimize external interference. 
    Where applicable, statistical tests are applied to support the evaluation of the experimental data.

  \subsection{Requirements} \label{subsec:requirements}
    To ensure a structured and reliable experimental evaluation of Rowhammer attacks on Raspberry Pi platforms, 
    a set of technical, and ethical requirements is defined. These requirements guide the design, 
    implementation, execution, and evaluation of the experimental framework and ensure that the results are 
    reproducible, and meaningful.

    \subsubsection{Functional requirements} \label{subsubsec:functional-requirements}
      The experimental setup must support a configurable system with adjustable parameters, including the 
      selection of different hammering techniques such as single-sided, double-sided, one-location, and if feasible 
      many-sided hammering. It must allow configuration of the number of iterations and hammer cycles, automatically detect 
      the memory page size, and support configurable initialization patterns, for example using \texttt{0} or \texttt{1}. The amount 
      of allocated \gls{ram} must be adjustable as a percentage of the available memory, and the framework must 
      incorporate bit masks and bank addressing functions. Automatic detection of hardware characteristics, 
      including the Raspberry Pi model, \gls{ram} size, and architecture, is required. Furthermore, the system must 
      support cache bypass mechanisms, virtual-to-physical-to-\gls{dram} address mapping, and, if feasible, mitigation 
      bypass techniques. Hammering operations must include bit flip detection and classification, such as 
      distinguishing between \texttt{0→1} and \texttt{1→0} transitions, as well as verification of detected bit flips. Experimental 
      output must be logged to the terminal, with all measurements recorded in JSON format and individual 
      measurements additionally stored in \texttt{.log} files.
      
    \subsubsection{Non-functional requirements} \label{subsubsec:non-functional-requirements}
      The implementation requires root privileges in user space to execute privileged instructions.
      All \hyperref[tab:experimental-setups]{Raspberry Pi 4 and Raspberry Pi 5 devices} must operate in ARMv8 (AArch64) mode. The 
      system must meet requirements regarding performance and efficiency and include robust error-handling mechanisms. 
      Reproducibility is essential, such that repeated measurements yield similar results under comparable conditions. Ethical 
      considerations are explicitly addressed by conducting all experiments in isolated environments to prevent 
      accidental harm.
