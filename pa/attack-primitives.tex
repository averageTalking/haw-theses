\section{Attack Primitives} \label{sec:attack-primitives}
  For Rowhammer attacks to be feasible, three fundamental prerequisites must be satisfied, commonly referred
  to as attack primitive~\cite{presRHAttacksWalkthrough, toolRowhammering}. These primitives define the necessary properties of memory accesses that 
  enable the electrical disturbance of \gls{dram} cells required for inducing bit flips.
  \vspace{0.5cm}
  \\
  \textbf{P1.} Uncached 
  \\
  \textbf{P2.} Targeted 
  \\
  \textbf{P3.} Fast
  \vspace{0.5cm}
  \\
  The first prerequisite, denoted as \textbf{\hyperref[sec:uncached]{P1. Uncached}}, requires that memory accesses bypass 
  the \gls{cpu} cache and reach the \gls{dram} directly. \gls{dram} cells are only electrically stressed when the memory controller opens 
  and closes the corresponding \gls{dram} row, which occurs exclusively on accesses that are not served by the cache. 
  Ensuring that each memory access physically toggles \gls{dram} cells, rather than being handled by the \gls{cpu} cache, 
  is therefore essential. This condition is critical because Rowhammer relies on high-frequency direct \gls{dram} 
  activations, which can only be achieved when cached accesses are avoided. If memory accesses remain cached, 
  the hammering speed is reduced by orders of magnitude and falls far below the threshold necessary to discharge 
  adjacent cells. Consequently, forcing cache misses or using uncached memory regions is mandatory to generate 
  the required activation frequency.
  \\
  The second prerequisite, \textbf{\hyperref[sec:targeted]{P2. Targeted}}, specifies that memory accesses must be directed to 
  precisely chosen physical \gls{dram} rows. To accomplish this, an attacker must determine how virtual addresses are mapped onto the 
  underlying physical \gls{dram} structures in order to identify aggressor rows that surround a specific victim row. 
  This process requires reconstructing the \gls{dram} address mapping scheme. Only with accurate knowledge of this 
  mapping can hammering be focused such that specific adjacent rows are repeatedly activated. Without correct 
  targeting, hammering becomes largely probabilistic and rarely, if ever, results in disturbances of victim data.
  \\
  The third prerequisite, \textbf{\hyperref[sec:fast]{P3. Fast}}, demands that memory accesses occur at a rate that outpaces 
  the \gls{dram} refresh mechanism. The attack fundamentally depends on hammering aggressor rows faster than refresh cycles can restore 
  the electrical charge in adjacent victim cells. Achieving this requires a high access throughput to the same 
  \gls{dram} rows with minimal delay between consecutive activations. In practice, this also implies that mitigation 
  mechanisms designed to limit activation frequency or to increase refresh activity - such as \gls{trr} 
  or an increased refresh rate - often need to be bypassed. Without overcoming such defenses, the hammering 
  intensity remains insufficient to reliably induce bit flips.
  