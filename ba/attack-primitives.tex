\section{Attack Primitives} \label{sec:attack-primitives}
  The feasibility of Rowhammer attacks depends on a set of fundamental attack primitives, which 
  define the conditions under which the attack can succeed [106, 58]. Specifically, three 
  properties of memory accesses must simultaneously hold.
  \vspace{0.5cm}
  \\
  \textbf{P1.} Uncached 
  \\
  \textbf{P2.} Targeted 
  \\
  \textbf{P3.} Fast
  \vspace{0.5cm}
  \\
  First, memory accesses must be \textbf{\hyperref[sec:uncached]{uncached}} in order to reach the \gls{dram} 
  directly, thereby bypassing 
  the \gls{cpu} cache. \gls{dram} cells are only electrically stressed when the memory controller opens and 
  closes an actual \gls{dram} row. Consequently, this primitive ensures that each memory access physically 
  toggles \gls{dram} cells instead of being served from the \gls{cpu} cache. This property is critical because 
  Rowhammer relies on high-frequency, direct \gls{dram} activations, which only occur when cache effects 
  are avoided. If memory accesses remain cached, the effective hammering speed decreases by orders 
  of magnitude and falls far below the threshold required to discharge charge in adjacent cells. 
  Therefore, forcing cache misses, or alternatively using uncached memory regions, is mandatory 
  to generate the activation frequency necessary for a successful attack.
  \\
  Second, memory accesses must be \textbf{\hyperref[sec:targeted]{targeted}} such that they reach a 
  specific intended physical \gls{dram} 
  row. This requires an understanding of \gls{dram} mapping, as the attacker must determine how virtual 
  addresses are translated to physical \gls{dram} structures in order to select aggressor rows that 
  surround a victim row. Achieving this goal necessitates reconstructing the \gls{dram} address scheme. 
  Only with accurate knowledge of this mapping can hammering be focused so that specific adjacent 
  rows are repeatedly activated. Without correct mapping, hammering becomes largely probabilistic 
  and rarely, if ever, results in disturbances to victim data.
  \\
  Third, memory accesses must be sufficiently \textbf{\hyperref[sec:fast]{fast}} to race against the 
  next \gls{dram} row refresh. The 
  attack depends on hammering memory rows at a higher rate than \gls{dram} refresh cycles can restore 
  charge in adjacent cells. This, in turn, requires achieving high access throughput to the same 
  \gls{dram} rows with minimal delay. To sustain such high-frequency activation, optimal access parameters 
  must be used, and the attack must often bypass mitigation mechanisms. In particular, defenses that 
  throttle activation patterns or increase refresh activity, such as \gls{trr} or 
  increased refresh rates, must frequently be circumvented to maintain sufficient hammering intensity.
  