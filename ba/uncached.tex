\section{Uncached} \label{sec:uncached}
    In the following, each attack primitive is discussed in turn, beginning with \hyperref[sec:attack-primitives]{P1. Uncached}, which 
    establishes the foundation for inducing repeated \gls{dram} row activations. The P1. Uncached primitive 
    is realized by preventing data from being stored in or retrieved from the cache, thereby ensuring 
    that all memory accesses are served directly from \gls{dram}. To this end, the available techniques for 
    bypassing the cache are first presented, followed by a focused discussion of two mechanisms that 
    are particularly relevant for Raspberry Pi platforms: cache eviction and cache flushing.
    \\
    To enable the memory access behavior required for triggering Rowhammer, several cache-bypass 
    techniques are considered [106, 58]. These techniques include the use of non-temporal memory 
    accesses, the allocation of uncached memory regions, cache eviction strategies, and explicit cache 
    flush operations. Prior work has shown that triggering Rowhammer on ARM-based devices presents 
    substantial challenges, as successful exploitation requires rapid, repeated access to identical 
    \gls{dram} locations [48]. \gls{cpu} caches significantly obstruct this process by intercepting repeated accesses, 
    thereby preventing them from reaching \gls{dram}. Consequently, mitigating cache effects becomes a central 
    prerequisite. While x86 architectures provide well-established mechanisms - such as cache flushing 
    via the \texttt{clflush} instruction, non-temporal accesses, and eviction-based techniques - [48] demonstrates 
    that these approaches are either impractical or not well suited for direct adaptation to ARM platforms, 
    thereby motivating the exploration of alternative, architecture-specific solutions. Since successful 
    Rowhammer attacks require high-frequency access to \gls{dram}, cache memory can significantly disrupt the 
    attack by absorbing repeated accesses and reducing the number of \gls{dram} activation's. Consequently, 
    uncached access can be achieved through two complementary strategies, either by explicitly removing 
    data from the cache or by preventing data from being cached in the first place [lecture-introduction].

    \subsection{Non-temporal Accesses} \label{subsec:non-temporal-accesses}
        Prior work introduces a Rowhammer triggering technique on x86 platforms that leverages \gls{nt} 
        memory instructions [105]. \gls{nt} accesses are intended for data with low temporal locality, i.e., 
        data that is unlikely to be reused in the near future and therefore should not occupy cache space, 
        such as streaming data. By design, these accesses can reduce or avoid cache involvement, which has 
        been shown to significantly facilitate Rowhammer attacks on x86 systems [105]. On ARMv8 architecture, 
        however, \gls{nt} load and store instructions (e.g., \texttt{LDNP} and \texttt{STNP}) merely act as advisory hints to the memory 
        system rather than enforcing uncached behavior [Drammer]. Empirical results indicate that accessed data 
        may still be cached in practice [48]. Moreover, multiple consecutive writes to the same address are 
        collected within a single write-combining buffer, with only the last write being committed to \gls{dram} [58].
        This aggregation substantially reduces the effective access frequency to \gls{dram}, rendering \gls{nt}-based approaches 
        unsuitable for achieving the activation rates required for Rowhammer attacks on Raspberry Pi devices.
    
    \subsection{Uncached Memory} \label{subsec:uncached-memory}
        Another approach involves allocating memory regions that are explicitly marked as uncached, for example 
        through the use of ION, such that every access is served directly from \gls{dram} without interacting with the 
        cache [38]. This method is particularly suitable for ARM-based mobile architectures, such as ARMv7 and 
        ARMv8, where cache flush instructions may require elevated privileges and non-temporal stores may still 
        be subject to caching in practice [58]. A key advantage of this technique is that it operates entirely 
        in unprivileged user space and does not require special permissions. Prior work describes how 
        user-accessible \gls{dma} buffers exposed by the Android ION memory management interface can be leveraged to 
        bypass \gls{cpu} caches, thereby enabling Rowhammer triggering from user space [38, 48]. However, subsequent 
        defenses, such as [GuardION], have rendered Rowhammer bugs triggered in this way non-exploitable. 
        Importantly, this technique is not applicable to Raspberry Pi systems, as the Android-specific ION framework 
        is not available in standard Linux-based Raspberry Pi environments, which prevents the allocation or control 
        of such uncached \gls{dma} buffers.
    
    \subsection{Cache eviction} \label{subsec:cache-eviction}
    This is a test.

    \subsubsection{Fundamentals} \label{subsubsec:fundamentals}
        This is a test.
    
    \subsubsection{Cache Attacks} \label{subsubsec:cache-attacks}
        This is a test.
    
    \subsubsection{Limitations} \label{subsubsec:limitations}
        This is a test.

    \subsection{Flush cache} \label{subsec:flush-cache}
    Flushing the cache explicitly invalidates a specific cache line so that the next memory access must be 
    served from \gls{dram} rather than from a cache level. The primary purpose of this operation is to force 
    direct \gls{dram} activation instead of cache hits, thereby guaranteeing a cache miss on every hammer iteration. 
    As a result, very high row-activation rates can be achieved, which are essential for effective Rowhammer 
    attacks. On x86 architectures, this behavior is implemented via the \texttt{clflush} instruction [5], whereas 
    ARM-based systems rely on cache flush maintenance instructions, such as \texttt{DC CIVAC}, \texttt{DC CVAC}, \texttt{DC ZVA}, and 
    related operations, to achieve comparable effects [48].
    \\
    In ARM terminology, the word flush is often used in descriptions of clean and invalidate 
    operations, although the architecture specification generally uses only the terms clean and invalidate 
    [72]. 
    \\
    \textbf{Invalidating} a cache or cache line means clearing it of data by clearing the valid bit of one 
    or more cache lines. The cache must always be invalidated after reset because its contents are undefined. 
    If the cache contains dirty data, simple invalidation is generally incorrect, since updated data from 
    write-back cacheable regions would be lost [109].
    \\
    \textbf{Cleaning} a cache or cache line refers to writing the contents of dirty cache lines back to main memory 
    and clearing the dirty bits, which makes the contents of the cache and main memory coherent with each 
    other [109]. This operation is only applicable to data caches that use a write-back policy. Cache 
    clean and invalidate operations can be performed by cache set or way, or by virtual address.
    \\
    The \textbf{\gls{poc}} is defined as the point at which all blocks that can access memory, such 
    as processor cores or DMA engines, are guaranteed to observe the same copy of a given address [109]. 
    Commonly, this point corresponds to the main external system memory, as illustrated in 
    \hyperref[fig:poc-pou]{Figure~\ref{fig:poc-pou}b}. 
    \\
    Closely related is the \textbf{\gls{pou}}, which denotes the point in the memory hierarchy at which the instruction cache, 
    data cache, and translation table walks of a core are guaranteed to observe the same copy of a memory location [109]. 
    In a system with Harvard L1 caches and a \gls{tlb} for caching translation table entries, a unified L2 cache represents 
    the Point of Unification. If no external cache is present, the main memory itself serves as the \gls{pou} [72], as 
    also shown in \hyperref[fig:poc-pou]{Figure~\ref{fig:poc-pou}a}.
    \begin{figure}[htbp]
        \centering
        \frame{\includegraphics[width=\linewidth]{img/poc-pou.pdf}}
        \caption{Point of Coherency and of Unification [109]}
        \label{fig:poc-pou}
    \end{figure}

    \subsubsection{Instruction Types} \label{subsubsec:instruction-types}
        The original Rowhammer study by Kim et al. [5] employed the \texttt{clflush} instruction to force cache lines 
        out of the cache hierarchy. \texttt{clflush} ensures that each memory access reaches physical \gls{dram}, thereby 
        enabling the high-frequency row activations required for Rowhammer. Raspberry Pi devices, however, 
        use ARM CPUs, which do not provide \texttt{clflush} or a direct equivalent. Instead, these systems rely on 
        ARM cache maintenance instructions to control the cache state. Although architecturally different, 
        these instructions fulfill the same fundamental requirement by bypassing or invalidating caches so 
        that memory accesses are actually served from \gls{dram}. ARM cache maintenance instructions are designed 
        to clean, invalidate, or clean and invalidate specific cache lines or entire cache structures. In 
        the context of Rowhammer attacks on ARM, these instructions are used to invalidate targeted cache 
        lines so that subsequent loads cause \gls{dram} accesses, to clean dirty lines when required so that writes 
        propagate to memory, and to maintain a stable state in which hammering produces real row activations 
        rather than being absorbed by the cache hierarchy. Overall, they are used to force memory accesses 
        to bypass or invalidate cache lines, thereby enabling direct \gls{dram} access.
        \\
        On Intel x86 systems, the \texttt{clflush} instruction invalidates the cache line containing the specified 
        address from all levels of the cache hierarchy, including both data and instruction caches 
        [72]. If a cache line at any level of the hierarchy is dirty, meaning 
        that it contains modified data [151], it is written back to memory before being invalidated 
        [150, 72]. The \texttt{clflush} instruction is not privileged and is therefore 
        available in userspace [150, 72], which makes it particularly suitable 
        for userspace Rowhammer implementations.
        \\
        On ARM architectures, the \texttt{DC CIVAC} instruction performs a clean and invalidate operation on a data 
        cache line by virtual address to the Point of Coherency. If access from Exception Level 0 (EL0) is 
        enabled, executing this instruction at EL0 may generate a permission fault. Its execution may also 
        require a translation from virtual to physical address, and it is defined as a 64-bit system 
        instruction [152, 109]. Similarly, the \texttt{DC CVAC} instruction cleans a data cache line by address to 
        the Point of Coherency. It is also a 64-bit system instruction and may generate a permission fault 
        when executed at EL0 if such access is enabled, and it may require address translation [153, 109]. 
        The \texttt{DC ZVA} instruction was introduced to efficiently zero out a block of memory [~Triggering RH]. 
        It zeroes a naturally aligned block of N bytes, and is 
        likewise a 64-bit system instruction [154, 109]. The ability to preload the data cache with zero 
        values using \texttt{DC ZVA} is new in ARMv8-A. Cache line zeroing behaves similarly to a prefetch by hinting 
        that certain addresses are likely to be used in the future; however, zeroing can be significantly 
        faster because it does not require waiting for external memory accesses to complete [~Triggering RH, Raspberry Pi 3B+].
    
    \subsubsection{Cache Behavior} \label{subsubsec:cache-behavior}
        \hyperref[tag:cache-bypass]{Table~\ref{tab:cache-bypass}} presents cache bypass measurements obtained under 
        different execution modes and measurement units 
        across several Raspberry Pi platforms. The table reports \gls{cpu} cycle counts for baseline measurements in 
        kernel and user space, as well as measurements obtained using a kernel module and explicit userspace 
        cache-bypass mechanisms using \texttt{DC CIVAC} und \texttt{STR}. In addition, userspace measurements are also 
        reported in nanoseconds. The nanosecond 
        values were measured using POSIX functions.
        \begin{table}[ht]
            \centering
            \begin{tabular}{llccccc}
                \hline
                Mode & Unit & RPi4 4GB & RPi4 8GB & RPi5 4GB & RPi5 8GB & RPi5 16GB \\
                \hline
                kernel baseline & Cycles & 78.71 & 78.16 & - & - & - \\
                kernel module & Cycles & 416.83 & 468.56 & - & - & - \\
                user baseline & Cycles & 80.04 & 88.76 & 43.19 & 43.21 & 43.21 \\
                userspace & Cycles & 431.76 & 581.20 & 489.60 & 483.14 & 495.90 \\
                userspace & ns & 341.92 & 334.51 & 263.49 & 248.34 & 249.67 \\
            \hline
            \end{tabular}
            \caption{Cache Bypass with a variety of measurements from userspace and as kernel module}
            \label{tab:cache-bypass}
        \end{table}
        The userspace measurements demonstrate a significant increase in \gls{cpu} cycles compared to the respective baselines. 
        On the RPi4, cache-bypassing measurements in userspace range between 431 and 581 cycles, compared to a baseline 
        of approximately 84 cycles. On the RPi5, userspace cache-bypassing measurements range between 416 and 468 cycles, 
        while the baseline remains at approximately 43 cycles. This clearly indicates a substantial increase in memory 
        access latency when the cache is bypassed.
        \\
        Kernel module measurements confirm a similar trend for the RPi4. Here, cache-bypassing measurements range from 
        432 to 581 \gls{cpu} cycles, compared to a baseline of approximately 78 cycles. For the RPi5, kernel module 
        measurements could not be interpreted due to the Zero Latency Readings Problem described in [Praxisarbeit], 
        which prevents meaningful analysis of the collected data.
        \\
        To statistically evaluate the observed differences, a \textit{Welch t-test} was conducted, yielding a test statistic 
        of \texttt{t = -67.70}. The \textit{Welch t-test} assesses whether the means of two independent samples differ significantly 
        without assuming equal variances. The result indicates a statistically significant difference between baseline 
        measurements without cache flushing and measurements obtained using \texttt{DC CIVAC + STR}. In addition, \textit{Cohen’s d} 
        was calculated as \texttt{d = -3.03}. \textit{Cohen’s d} is a standardized measure of effect size that expresses the magnitude 
        of the difference between two means relative to their pooled standard deviation. The obtained value represents 
        a very large effect size, confirming a strong impact of the cache flush operation on memory access latency.
        \\
        Overall, the results show that the applied cache-bypass mode effectively bypasses the cache hierarchy, 
        leading to an approximately four- to six-fold increase in observed latency on the RPi4 and an approximately 
        five- to six-fold increase on the RPi5. The applied statistical tests support the conclusion that this 
        increase is highly significant and cannot be attributed to random variation. Consequently, \texttt{DC CIVAC + STR} 
        can be considered a robust cache flush operation for inducing measurable memory access delays in both userspace 
        and kernel space, with results that are consistent across multiple devices and measurement interfaces.

