\section{Fast} \label{sec:fast}
    This is a test.

    % \subsection{Mitigation's} \label{subsec:mitigations}
    %     % In General
    %     The Rowhammer vulnerability has been demonstrated in a wide range of environments, including sandboxed 
    %     systems, native platforms, virtual machines, JavaScript contexts, mobile devices, and even across 
    %     networks [53]. Mitigation strategies aim to reduce both the likelihood and the impact of Rowhammer-induced 
    %     disturbance errors by altering system behavior, memory management policies, or \gls{dram} operation. Numerous 
    %     mitigation techniques have been proposed to address the Rowhammer problem in general [Quote of ~Triggering RH]. 
    %     These approaches can be broadly classified into two categories: software-based protections and 
    %     hardware-based protections [36, 38].
    %     \\
    %     % Software-Based Defenses
    %     \textbf{Software-based defenses} against Rowhammer attacks are often complex, costly, and may remain incomplete [37]. 
    %     These approaches can be further divided into prevention or restriction techniques and detection strategies.
    %     \textbf{Prevention and restriction approaches} aim to constrain or disable memory access patterns and instructions 
    %     that can lead to high activation rates, thereby preventing Rowhammer conditions from occurring. Examples 
    %     include \textit{MASCAT}, which performs static analysis of binaries to detect suspicious instruction sequences, 
    %     though this approach faces open problems such as false positives. \textit{Instruction blacklisting} restricts access 
    %     to specific commands, such as cache flush instructions, to limit potential exploitation. Additionally, 
    %     \textit{B-CATT} disables vulnerable physical memory or \gls{dram} regions to prevent Rowhammer-induced errors [43, 101].
    %     \textbf{Detection approaches} focus on identifying behaviors indicative of Rowhammer activity. This includes monitoring 
    %     abnormal activation frequencies or microarchitectural patterns so that countermeasures can be triggered before 
    %     bit flips occur. Rowhammer attacks are characterized by a significantly higher number of cache misses and hits 
    %     compared to other attack types. Detection systems also aim to identify hammering patterns, frequent row 
    %     activations, high cache-miss rates (as in \textit{ANVIL}), and other suspicious instruction sequences.
    %     \\
    %     % Hardware-Based Defenses
    %     The root cause of Rowhammer stems from disturbance errors arising from physical interactions between multiple 
    %     cells in highly scaled \gls{dram}. Increased integration density and smaller cell geometries reduce noise margins 
    %     and exacerbate charge leakage effects. A straightforward hardware-based remedy would involve reducing \gls{dram} 
    %     cell density or increasing the physical cell size; however, this would require higher voltages and larger 
    %     charge reservoirs, resulting in unacceptable power and performance costs. Lowering integration density to 
    %     mitigate interference also significantly reduces memory capacity, creating a fundamental trade-off between 
    %     security and storage density. Since commercial \gls{dram} must maximize usable capacity, the underlying physical 
    %     disturbance mechanism cannot be entirely eliminated in current technology nodes. Moreover, \textbf{hardware-based 
    %     approaches} may introduce additional latency, increase energy consumption, or raise implementation complexity.
    %     \textbf{\gls{ecc}} can correct single-bit errors using stored parity information. However, 
    %     \gls{ecc} can only detect and correct single-bit flips and is insufficient for addressing multi-bit flips, which 
    %     have become increasingly common [~Triggering RH].
    %     \textbf{Increasing the \gls{dram} refresh rate} represents another mitigation strategy. \gls{dram} cells are volatile and 
    %     require refreshing at standard intervals of 32 ms to 64 ms [~Triggering RH]. The simplest approach is to increase 
    %     the refresh rate by two to four times [57, 34], which reduces charge leakage exposure. However, higher refresh 
    %     frequencies delay data access, negatively impacting system performance [34]. While this strategy makes Rowhammer 
    %     attacks more difficult, it is often unacceptable due to its detrimental effects on performance and energy efficiency 
    %     [~Triggering RH].
    %     \textbf{\gls{trr}} detects frequently activated rows and refreshes adjacent rows to prevent 
    %     disturbance errors. Nonetheless, \textit{\gls{trr}} is vendor-specific and non-mandatory, and it is vulnerable to bypasses such 
    %     as \textit{TRRespass} and \textit{Halfdouble} due to imperfect identification of victim rows.
    %     \\
    %     % Cross-Cutting Limitations
    %     Several limitations affect mitigation strategies across both software and hardware approaches. Certain hammering 
    %     techniques can avoid cache misses, rendering detection ineffective. Detection heuristics frequently generate high 
    %     false-positive rates in real workloads. Mitigations may restrict system functionality or break compatibility with 
    %     existing software. Single-error correction remains insufficient, leaving multi-bit flips exploitable. Hardware 
    %     protections are often manufacturer-dependent (e.g., \textit{\gls{trr}}) and may degrade performance, increase energy consumption, 
    %     or be difficult to deploy on existing platforms, such as ARM architectures [~Triggering RH]. Software-based approaches 
    %     generally provide only partial workarounds and cannot fully overcome the physical limitations of \gls{dram}, often remaining 
    %     incomplete. Incorrect identification of victim rows can be exploited, as demonstrated by \textit{TRRespass} and \textit{Halfdouble} 
    %     attacks.
    %     \\
    %     % Conclusion
    %     To date, existing mitigation strategies only reduce the risk of Rowhammer attacks; no solution provides 
    %     comprehensive protection across all DRAM generations and attack variants. Full protection would require 
    %     reliably tracking or constraining every row activation at the hardware level, independent of access patterns, 
    %     instruction choices, or DRAM vendor behavior, to ensure that no combination of activations could bring 
    %     any victim cell close to its disturbance threshold.

    % \subsection{TRR Bypass} \label{subsec:trr-bypass}
    %     This is a test.

    % \subsection{Optimization} \label{subsec:optimization}
    %     This is a test.
        